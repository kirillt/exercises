\documentclass[10pt,pdf,hyperref={unicode}]{beamer}
\usepackage[english,russian]{babel}
\usepackage[utf8x]{inputenc}
\usepackage{graphicx}
\usepackage{amsmath}
\usepackage{agda}
\usepackage{ucs}

\usetheme{Boadilla}

   \begin{document}

      \title[Proof assistants и семантики]{Использование proof assistants для описания операционных семантик}
      \author[Кирилл Таран]{Кирилл Таран}
      \institute{445 группа}

      \begin{frame}
         \titlepage
         \hfill науч. рук.: Булычев Д.Ю.
      \end{frame}

      \section{О проекте}

      \begin{frame}
         \frametitle{О проекте}
         \begin{itemize}
            \item Исследование в рамках лаборатории JetBrains
            \item Области исследования:
               \begin{itemize}
                  \item разработка языков программирования
                  \item сертификационное программирование
                     \begin{itemize}
                     \item К программе прилагаются ``сертификаты'' -- формальные описания
                      и доказательства её полезных свойств.
                     \end{itemize}
               \end{itemize}
         \end{itemize}
      \end{frame}

      \section{Введение в Proof assistants}

      \begin{frame}
         \frametitle{Введение}
         \framesubtitle{Proof assistants}
         \begin{itemize}
            \item Сертификационное программирование появилось
            в связи с развитием т.н. proof assistants -- систем
            для интерактивного построения доказательств теорем,
            автоматической проверки доказательств и т.д.
            \item В данной работе рассмотрены системы,
            основанные на зависимых типах
         \end{itemize}
      \end{frame}

      \begin{frame}
         \frametitle{Введение}
         \framesubtitle{Зависимые типы}
         \begin{itemize}
            \item ``Зависимые типы'' -- могут быть параметризованы термами (значениями),
             в отличие от простых полиморфных типов, которые могут быть
             параметризованы только другими типами
            \item Язык программирования с зависимыми типами позволяет работать
             с теоремами (Curry--Howard correspondence):
               \begin{itemize}
                  \item формулировка теоремы -- тип функции
                  \item терм, определяющий функцию -- доказательство
               \end{itemize}
         \end{itemize}
      \end{frame}

      \section{Цели работы}

      \begin{frame}
         \frametitle{Цели работы}
         \begin{itemize}
            \item Операционная семантика -- дедуктивная система, поэтому
             инструменты для доказательства теорем могут оказаться полезными при работе с ней
            \item Цель курсовой -- оценить применимость p.a. к задачам разработки языков программирования
            \item Предполагаемые результаты:
            \begin{itemize}
               \item автоматическая генерация интерпретатора по операционной семантике
               \item доказательство каких-либо свойств об интерпретации (по возможности автоматическое)
               \item рассмотреть различные системы (по возможности)
            \end{itemize}
         \end{itemize}
      \end{frame}

      \section{Основная часть}

      \newcommand{\control}[2]{\langle #1, #2 \rangle}
      \newcommand{\rulens}[3]{\control{#1}{#2} \to #3}
      \newcommand{\ruless}[4]{\rulens{#1}{#2}{#3} \frac{#4}{}}
      \newcommand{\tree}[3]{\cfrac{#1 \hspace{10mm} #2}{#3}}

      \begin{frame}
         \frametitle{Операционные семантики}
         \framesubtitle{Семантическое дерево}
         \begin{itemize}
            \item Семантика задаёт переходы между состояниями
            \item Программе соответствует семантическое дерево --
             дерево применений правил семантики к синтаксическим
             элементам программы
             \item Пример дерева:
             \begin{itemize}
                \item[] программа `(z:=x ; x:=y) ; y:=z'
                \newline
                \item[] $\tree{\tree{\rulens{z:=x}{s_0}{s_1}}
                                    {\rulens{x:=y}{s_1}{s_2}}
                                    {\rulens{z:=x;x:=y}{s_0}{s_2}}}
                              {\rulens{y:= z}{s_2}{s_3}}
                              {\rulens{(z:=x ; x:=y) ; y:=z}{s_0}{s_3}}$
                \newline
                \item[] $s_0$, $s_1$, $s_2$ -- состояния программы (списки значений)
             \end{itemize}
         \end{itemize}
      \end{frame}


      \newcommand{\seqns}[0]{\tree{\rulens{p_1}{s_0}{s_1}}{\rulens{p_2}{s_1}{s_2}}{\rulens{p_1 ; p_2}{s_0}{s_2}}}
      \newcommand{\iftruens}[0]{s_0[b] = true\Rightarrow\tree{\rulens{p_1}{s_0}{s_1}}{}
                                                             {\rulens{\mbox{if b then $p_1$ else $p_2$}}{s_0}{s_1}}}
      \newcommand{\assignns}[0]{\rulens{x:=a}{s}{s[x \mapsto a]}}


      \begin{frame}
         \frametitle{Операционные семантики}
         \framesubtitle{Правила вывода (big-step/natural semantics)}

         \begin{align*}
            \begin{tabular}{l || r}
               ``;''       & $\seqns$    \\
               \\ \hline \\
               ``if-true'' & $\iftruens$ \\
               \\ \hline \\
               ``:=''      & $\assignns$ \\
            \end{tabular}
         \end{align*}
      \end{frame}

      \begin{frame}
         \frametitle{Операционная семантика на Agda}
         \framesubtitle{Тип данных}
         \begin{itemize}
            \item Семантика представляется на языке Agda в виде типа данных:
            \begin{itemize}
               \item Термы типа -- семантические деревья
               \item Параметры типа -- входные/выходные состояния перехода и программа
               \item Конструкторы -- правила перехода
            \end{itemize}
            \item[] \small{\begin{code}
\>[2]\AgdaKeyword{data} \AgdaDatatype{Transition} \AgdaSymbol{(}\AgdaBound{s₁} \AgdaSymbol{:} \AgdaFunction{State}\AgdaSymbol{)} \AgdaSymbol{:} \AgdaDatatype{S} \AgdaSymbol{→} \AgdaFunction{State} \AgdaSymbol{→} \AgdaPrimitiveType{Set} \AgdaKeyword{where}\<%
\\
\>[2]\AgdaIndent{4}{}\<[4]%
\>[4]\AgdaInductiveConstructor{[skip]} \<[13]%
\>[13]\AgdaSymbol{:} \<[27]%
\>[27]\AgdaDatatype{Transition} \AgdaBound{s₁} \<[43]%
\>[43]\AgdaInductiveConstructor{skip} \<[76]%
\>[76]\AgdaBound{s₁}\<%
\\
\>[2]\AgdaIndent{4}{}\<[4]%
\>[4]\AgdaInductiveConstructor{[assign]}\AgdaIndent{3}{} \AgdaSymbol{:} \AgdaSymbol{∀} \AgdaSymbol{\{}\AgdaBound{k} \AgdaBound{v} \AgdaSymbol{\}} \AgdaSymbol{→} \AgdaDatatype{Transition} \AgdaBound{s₁} \AgdaSymbol{(}\AgdaInductiveConstructor{assign} \AgdaBound{k} \AgdaBound{v}\AgdaSymbol{)} \AgdaSymbol{((}\AgdaBound{k} \AgdaInductiveConstructor{,} \AgdaBound{s₁} \AgdaFunction{[[} \AgdaBound{v} \AgdaFunction{]]}\AgdaSymbol{)} \AgdaFunction{|>} \AgdaBound{s₁}\AgdaSymbol{)}\<%
\\
\>[2]\AgdaIndent{4}{}\<[4]%
\>[4]\AgdaInductiveConstructor{[comp]} \<[13]%
\>[13]\AgdaSymbol{:} \AgdaSymbol{∀} \AgdaSymbol{\{}\AgdaBound{s₂} \AgdaBound{s₃} \AgdaBound{p₁} \AgdaBound{p₂}\AgdaSymbol{\}} \AgdaSymbol{→} \AgdaDatatype{Transition} \AgdaBound{s₁} \<[53]%
\>[53]\AgdaBound{p₁} \<[60]%
\>[60]\AgdaBound{s₂}\<%
\\
\>[4]\AgdaIndent{34}{}\<[34]%
\>[34]\hspace{33mm}\AgdaSymbol{→} \AgdaDatatype{Transition} \AgdaBound{s₂} \<[56]%
\>[56]\AgdaBound{p₂} \<[63]%
\>[63]\AgdaBound{s₃}\<%
\\
\>[4]\AgdaIndent{33}{}\<[34]%
\>[34]\hspace{33mm}\AgdaSymbol{→} \AgdaDatatype{Transition} \AgdaBound{s₁} \AgdaSymbol{(}\AgdaInductiveConstructor{comp} \AgdaBound{p₁} \AgdaBound{p₂}\AgdaSymbol{)} \AgdaBound{s₃}\<%
\\
\>[2]\AgdaComment{-\-- et cetera ...}\<%
\end{code}
}
         \end{itemize}
      \end{frame}

      \begin{frame}
         \frametitle{Операционная семантика на Agda}
         \framesubtitle{Тип данных}
         \begin{itemize}
            \item Преимущества подобного представления:
            \begin{itemize}
               \item Задаёт отношение ``вычислимости'', которое можно использовать в типах функций
               \item Кроме результата вычисления интерпретатор будет строить дерево вычисления,
                     которое можно использовать для анализа интерпретатора
               \item Интерпретатор можно попытаться вывести автоматически
            \end{itemize}
            \item[] \small{\begin{code}
\>[2]\AgdaKeyword{data} \AgdaDatatype{Transition} \AgdaSymbol{(}\AgdaBound{s₁} \AgdaSymbol{:} \AgdaFunction{State}\AgdaSymbol{)} \AgdaSymbol{:} \AgdaDatatype{S} \AgdaSymbol{→} \AgdaFunction{State} \AgdaSymbol{→} \AgdaPrimitiveType{Set} \AgdaKeyword{where}\<%
\\
\>[2]\AgdaIndent{4}{}\<[4]%
\>[4]\AgdaInductiveConstructor{[skip]} \<[13]%
\>[13]\AgdaSymbol{:} \<[27]%
\>[27]\AgdaDatatype{Transition} \AgdaBound{s₁} \<[43]%
\>[43]\AgdaInductiveConstructor{skip} \<[76]%
\>[76]\AgdaBound{s₁}\<%
\\
\>[2]\AgdaIndent{4}{}\<[4]%
\>[4]\AgdaInductiveConstructor{[assign]}\AgdaIndent{3}{} \AgdaSymbol{:} \AgdaSymbol{∀} \AgdaSymbol{\{}\AgdaBound{k} \AgdaBound{v} \AgdaSymbol{\}} \AgdaSymbol{→} \AgdaDatatype{Transition} \AgdaBound{s₁} \AgdaSymbol{(}\AgdaInductiveConstructor{assign} \AgdaBound{k} \AgdaBound{v}\AgdaSymbol{)} \AgdaSymbol{((}\AgdaBound{k} \AgdaInductiveConstructor{,} \AgdaBound{s₁} \AgdaFunction{[[} \AgdaBound{v} \AgdaFunction{]]}\AgdaSymbol{)} \AgdaFunction{|>} \AgdaBound{s₁}\AgdaSymbol{)}\<%
\\
\>[2]\AgdaIndent{4}{}\<[4]%
\>[4]\AgdaInductiveConstructor{[comp]} \<[13]%
\>[13]\AgdaSymbol{:} \AgdaSymbol{∀} \AgdaSymbol{\{}\AgdaBound{s₂} \AgdaBound{s₃} \AgdaBound{p₁} \AgdaBound{p₂}\AgdaSymbol{\}} \AgdaSymbol{→} \AgdaDatatype{Transition} \AgdaBound{s₁} \<[53]%
\>[53]\AgdaBound{p₁} \<[60]%
\>[60]\AgdaBound{s₂}\<%
\\
\>[4]\AgdaIndent{34}{}\<[34]%
\>[34]\hspace{33mm}\AgdaSymbol{→} \AgdaDatatype{Transition} \AgdaBound{s₂} \<[56]%
\>[56]\AgdaBound{p₂} \<[63]%
\>[63]\AgdaBound{s₃}\<%
\\
\>[4]\AgdaIndent{33}{}\<[34]%
\>[34]\hspace{33mm}\AgdaSymbol{→} \AgdaDatatype{Transition} \AgdaBound{s₁} \AgdaSymbol{(}\AgdaInductiveConstructor{comp} \AgdaBound{p₁} \AgdaBound{p₂}\AgdaSymbol{)} \AgdaBound{s₃}\<%
\\
\>[2]\AgdaComment{-\-- et cetera ...}\<%
\end{code}
}
         \end{itemize}
      \end{frame}

      \begin{frame}
         \frametitle{Операционная семантика на Agda}
         \framesubtitle{Интерактивное построение интерпретатора}
            \begin{itemize}
               \item Код, введённый вручную:
               \item[] \footnotesize{\begin{code}\>\<%
\>[0]\AgdaIndent{2}{}\<[2]%
\>[2]\AgdaComment{-\-- Interpretation $s_1$ p = ($s_2$ : State, Transition $s_1$ p $s_2$)}\<%
\\
\>[0]\AgdaIndent{2}{}\<[2]%
\>[2]\AgdaFunction{interpret} \AgdaSymbol{:} \AgdaSymbol{(}\AgdaBound{s} \AgdaSymbol{:} \AgdaFunction{State}\AgdaSymbol{)} \AgdaSymbol{→} \AgdaSymbol{(}\AgdaBound{p} \AgdaSymbol{:} \AgdaDatatype{S}\AgdaSymbol{)} \AgdaSymbol{→} \AgdaRecord{Interpretation} \AgdaBound{s} \AgdaBound{p}\<%
\\
\\
\>[0]\AgdaIndent{2}{}\<[2]%
\>[2]\AgdaComment{-\-- \{!!\} -- ``hole'', выводится автоматически}\<%
\\
\>[0]\AgdaIndent{2}{}\<[2]%
\>[2]\AgdaFunction{interpret} \AgdaBound{s₁} \AgdaSymbol{(}\AgdaInductiveConstructor{skip} \<[31]%
\>[31]\AgdaSymbol{)} \AgdaSymbol{=} \AgdaSymbol{\{!!\}}\<%
\\
\>[0]\AgdaIndent{2}{}\<[2]%
\>[2]\AgdaFunction{interpret} \AgdaBound{s₁} \AgdaSymbol{(}\AgdaInductiveConstructor{assign} \<[25]%
\>[25]\AgdaBound{k} \<[29]%
\>[29]\AgdaBound{v} \AgdaSymbol{)} \AgdaSymbol{=} \AgdaSymbol{\{!!\}}\<%
\\
\>[0]\AgdaIndent{2}{}\<[2]%
\>[2]\AgdaFunction{interpret} \AgdaBound{s₁} \AgdaSymbol{(}\AgdaInductiveConstructor{comp} \<[25]%
\>[25]\AgdaBound{p₁} \AgdaBound{p₂}\AgdaSymbol{)} \AgdaKeyword{with} \AgdaFunction{interpret} \AgdaBound{s₁} \AgdaBound{p₁}\<%
\\
\>[0]\AgdaIndent{2}{}\<[2]%
\>[2]\AgdaSymbol{...} \AgdaSymbol{|} \AgdaInductiveConstructor{I} \AgdaBound{s₂} \AgdaInductiveConstructor{,} \AgdaBound{tr₁} \AgdaInductiveConstructor{I} \AgdaKeyword{with} \AgdaFunction{interpret} \AgdaBound{s₂} \AgdaBound{p₂}\<%
\\
\>[0]\AgdaIndent{2}{}\<[2]%
\>[2]\AgdaSymbol{...} \AgdaSymbol{|} \AgdaInductiveConstructor{I} \AgdaBound{s₃} \AgdaInductiveConstructor{,} \AgdaBound{tr₂} \AgdaInductiveConstructor{I} \AgdaSymbol{=} \AgdaSymbol{\{!!\}}\<%
\\
\>[0]\AgdaIndent{2}{}\<[2]%
\>[2]\AgdaComment{-\-- et cetera ...}\<%
\<\end{code}
}
            \end{itemize} %hack
            \begin{itemize}
               \item Сгенерированный код:
               \item[] \footnotesize{\begin{code}\>\<%
\>[0]\AgdaIndent{2}{}\<[2]%
\>[2]\AgdaComment{-\-- Interpretation $s_1$ p = ($s_2$ : State, Transition $s_1$ p $s_2$)}\<%
\\
\>[0]\AgdaIndent{2}{}\<[2]%
\>[2]\AgdaFunction{interpret} \AgdaSymbol{:} \AgdaSymbol{(}\AgdaBound{s} \AgdaSymbol{:} \AgdaFunction{State}\AgdaSymbol{)} \AgdaSymbol{→} \AgdaSymbol{(}\AgdaBound{p} \AgdaSymbol{:} \AgdaDatatype{S}\AgdaSymbol{)} \AgdaSymbol{→} \AgdaRecord{Interpretation} \AgdaBound{s} \AgdaBound{p}\<%
\\
\>[0]\AgdaIndent{2}{}\<[2]%
\>[2]\AgdaFunction{interpret} \AgdaBound{s₁} \AgdaSymbol{(}\AgdaInductiveConstructor{skip} \<[31]%
\>[31]\AgdaSymbol{)} \AgdaSymbol{=} \AgdaInductiveConstructor{I} \AgdaBound{s₁} \AgdaInductiveConstructor{,} \AgdaInductiveConstructor{[skip]} \AgdaInductiveConstructor{I}\<%
\\
\>[0]\AgdaIndent{2}{}\<[2]%
\>[2]\AgdaFunction{interpret} \AgdaBound{s₁} \AgdaSymbol{(}\AgdaInductiveConstructor{assign} \<[25]%
\>[25]\AgdaBound{k} \<[29]%
\>[29]\AgdaBound{v} \AgdaSymbol{)} \AgdaSymbol{=} \AgdaInductiveConstructor{I} \AgdaSymbol{(}\AgdaBound{k} \AgdaInductiveConstructor{,} \AgdaSymbol{(}\AgdaBound{s₁} \AgdaFunction{[[} \AgdaBound{v} \AgdaFunction{]]}\AgdaSymbol{))} \AgdaFunction{|>} \AgdaBound{s₁} \AgdaInductiveConstructor{,} \AgdaInductiveConstructor{[assign]} \AgdaInductiveConstructor{I}\<%
\\
\>[0]\AgdaIndent{2}{}\<[2]%
\>[2]\AgdaFunction{interpret} \AgdaBound{s₁} \AgdaSymbol{(}\AgdaInductiveConstructor{comp} \<[25]%
\>[25]\AgdaBound{p₁} \AgdaBound{p₂}\AgdaSymbol{)} \AgdaKeyword{with} \AgdaFunction{interpret} \AgdaBound{s₁} \AgdaBound{p₁}\<%
\\
\>[0]\AgdaIndent{2}{}\<[2]%
\>[2]\AgdaSymbol{...} \AgdaSymbol{|} \AgdaInductiveConstructor{I} \AgdaBound{s₂} \AgdaInductiveConstructor{,} \AgdaBound{tr₁} \AgdaInductiveConstructor{I} \AgdaKeyword{with} \AgdaFunction{interpret} \AgdaBound{s₂} \AgdaBound{p₂}\<%
\\
\>[0]\AgdaIndent{2}{}\<[2]%
\>[2]\AgdaSymbol{...} \AgdaSymbol{|} \AgdaInductiveConstructor{I} \AgdaBound{s₃} \AgdaInductiveConstructor{,} \AgdaBound{tr₂} \AgdaInductiveConstructor{I} \AgdaSymbol{=} \AgdaInductiveConstructor{I} \AgdaBound{s₃} \AgdaInductiveConstructor{,} \AgdaInductiveConstructor{[comp]} \AgdaBound{tr₁} \AgdaBound{tr₂} \AgdaInductiveConstructor{I}\<%
\\
\>[0]\AgdaIndent{2}{}\<[2]%
\>[2]\AgdaComment{-\-- et cetera ...}\<%
\<\end{code}
}
            \end{itemize}
      \end{frame}

      \begin{frame}
         \frametitle{Операционная семантика на Agda}
         \framesubtitle{Интерактивное построение интерпретатора}
            \begin{itemize}
               \item Автоматический вывод термов решается с помощью Agsy --
                компонента Agda для поиска type inhabitant (auto proof search)
               \item Agsy пытается разобрать по случаям аргументы функции и применить
                данные ему подсказки (леммы), при этом проверяя типизируемость генерируемого выражения
               \item К сожалению, есть ограничения реализации:
                  \begin{itemize}
                     \item сопоставление с образцом произвольных подвыражений
                      Agsy выводить не умеет -- пока что приходится писать вручную
                     \item но возможно получится обойтись функциями специального вида
                      для разбора случаев
                  \end{itemize}
            \end{itemize}
      \end{frame}

      \newcommand{\seqssn}[0]{\tree{\rulens{p_1}{s_0}{s_1}}{}{\ruless{p_1 ; p_2}{s_0}{s_1}{p_2}}}
      \newcommand{\seqssj}[0]{\tree{\ruless{p_1}{s_0}{s_1}{p_1'}}{}
                                   {\ruless{p_1 ; p_2}{s_0}{s_1}{p_1' ; p_2}}}
      \newcommand{\iftruess}[0]{s[b] = true\Rightarrow\ruless{\mbox{if b then $p_1$ else $p_2$}}{s}{s}{p_1}}

      \begin{frame}
         \frametitle{Операционные семантики}
         \framesubtitle{Правила вывода (small-step/structural semantics)}

         \begin{align*}
            \begin{tabular}{c || c}
               ``;''       & \begin{tabular}{c}
                                $\seqssn$ \\
                                \\
                                $\seqssj$
                             \end{tabular} \\
               \\ \hline \\
               ``if-true'' & $\iftruess$ \\
               \\ \hline \\
               ``:=''      & $\assignns$ \\
            \end{tabular}
         \end{align*}
      \end{frame}

      \begin{frame}
         \frametitle{Операционная семантика на Agda}
         \framesubtitle{Интерактивное доказательство свойств}
            \begin{itemize}
               \item Теорема об эквивалентности big-step и small-step семантик:
               \item[] \small{\begin{code}\>\<%
\>[0]\AgdaIndent{2}{}\<[2]%
\>[2]\AgdaFunction{sos≡ns} \AgdaSymbol{:} \AgdaSymbol{(}\AgdaBound{s} \AgdaSymbol{:} \AgdaFunction{State}\AgdaSymbol{)} \AgdaSymbol{→} \AgdaSymbol{(}\AgdaBound{p} \AgdaSymbol{:} \AgdaDatatype{S}\AgdaSymbol{)} \AgdaSymbol{→} \AgdaSymbol{∀} \AgdaSymbol{\{}\AgdaBound{s-sos} \AgdaBound{s-ns}\AgdaSymbol{\}}\<%
\\
\>[2]\AgdaIndent{4}{}\<[4]%
\>[4]\AgdaSymbol{→} \AgdaSymbol{(}\AgdaBound{tr-sos} \AgdaSymbol{:} \AgdaDatatype{Transition-SS*} \AgdaBound{s} \AgdaBound{p} \AgdaBound{s-sos}\AgdaSymbol{)}\<%
\\
\>[2]\AgdaIndent{4}{}\<[4]%
\>[4]\AgdaSymbol{→} \AgdaSymbol{(}\AgdaBound{tr-ns} \AgdaSymbol{:} \AgdaDatatype{Transition-NS} \AgdaBound{s} \AgdaBound{p} \AgdaBound{s-ns}\AgdaSymbol{)}\<%
\\
\>[2]\AgdaIndent{4}{}\<[4]%
\>[4]\AgdaSymbol{→} \AgdaBound{s-sos} \AgdaDatatype{≡} \AgdaBound{s-ns}\<%
\\
\>[0]\AgdaIndent{2}{}\<[2]%
\>[2]\AgdaFunction{sos≡ns} \AgdaBound{s} \AgdaBound{p} \AgdaSymbol{\{}\AgdaBound{s-sos}\AgdaSymbol{\}} \AgdaSymbol{\{}\AgdaBound{s-ns}\AgdaSymbol{\}} \AgdaBound{tr-sos} \AgdaBound{tr-ns} \AgdaSymbol{=} \AgdaSymbol{\{!!\}}\<%
\<\end{code}
}
               \item Кода в доказательстве очень много, без автоматического вывода тяжело
            \end{itemize}
      \end{frame}

      \section{Результаты}

      \begin{frame}
          \frametitle{Результаты}
          \begin{itemize}
             \item Найден способ полуавтоматического построения интерпретатора и доказательств свойств о нём
             \item Есть возможность дополнить существующие инструменты до полной автоматизации
             \item Были также рассмотрены Coq и Idris, использующие тактики вместо auto proof search
                \begin{itemize}
                   \item[] $\oplus$  тактики дают больше возможностей по управлению выводом
                   \item[] $\ominus$ но требуют работы по написанию proof script
                \end{itemize}
             \item Репозиторий с кодом:
                \begin{itemize}
                   \item https://github.com/kirillt/exercises/tree/master/semantics
                   \item[] (наброски семантик на Coq, Agda, Idris + интерпретатор на Agda)
                \end{itemize}
          \end{itemize}
      \end{frame}

  \end{document}
