\documentclass[12pt,pdf,hyperref={unicode}]{beamer}
\usepackage[english,russian]{babel}
\usepackage[utf8]{inputenc}
\usetheme{Boadilla}

   \begin{document}

      \title[Proof assistants и семантики]{Использование proof assistants для описания операционных семантик}
      \author[Кирилл Таран]{Кирилл Таран\newlineнауч. рук.: Булычев Д.Ю.}
      \institute{445 группа}

      \frame{\titlepage}

      \section{О проекте}
      \begin{frame}
         \frametitle{О проекте}
         \begin{itemize}
            \item Исследование в рамках лаборатории JetBrains
            \item Области исследования:
               \begin{itemize}
                  \item разработка языков программирования
                  \item сертификационное программирование
                     \begin{itemize}
                     \item К программе прилагаются ``сертификаты'' -- формальные описания
                      и доказательства её полезных свойств.
                     \end{itemize}
               \end{itemize}
         \end{itemize}
      \end{frame}

      % Сначала расскажу о самой технологии, а потом только раскрою цель работы.

      \section{Proof assistants}
      \begin{frame}
         \frametitle{Proof assistants}
         \begin{itemize}
            \item Сертификационное программирование появилось
            в связи с развитием т.н. proof assistants -- систем
            для интерактивного построения доказательств теорем,
            автоматической проверки доказательств и т.д.
            \item В данной работе рассмотрены системы,
            основанные на зависимых типах
         \end{itemize}
      \end{frame}

      \begin{frame}
         \frametitle{Зависимые типы}
         \begin{itemize}
            \item ``Зависимые типы'' -- могут быть параметризованы термами (значениями),
             в отличие от простых полиморфных типов, которые могут быть
             параметризованы только другими типами
            \item Язык программирования с зависимыми типами позволяет работать
             с теоремами (Curry--Howard correspondence):
               \begin{itemize}
                  \item формулировка теоремы -- тип функции
                  \item терм, определяющий функцию -- доказательство
               \end{itemize}
         \end{itemize}
      \end{frame}

      \section{Цели работы}
      \begin{frame}
         \frametitle{Цели работы}
         \begin{itemize}
            \item Операционная семантика -- дедуктивная система, поэтому
             системы доказательств теорем могут оказаться полезными при работе с ней
            \item Цель курсовой -- оценить применимость p.a. к задачам разработки языков программирования
            \item Предполагаемые результаты:
            \begin{itemize}
               \item автоматическая генерация интерпретатора по операционной семантике
               \item доказательство каких-либо свойств об интерпретации (по возможности автоматическое)
               \item рассмотреть различные системы (по возможности)
            \end{itemize}
         \end{itemize}
      \end{frame}

      \section{Операционные семантики}
      \begin{frame}
         \frametitle{Операционные семантики}
         \begin{itemize}
         \end{itemize}
      \end{frame}

  \end{document}
