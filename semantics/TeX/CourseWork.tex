\documentclass[aps,12pt,final,oneside,onecolumn,musixtex,superscriptaddress,centertags]{article} 
\usepackage[colorlinks=true,linkcolor=blue,unicode=true]{hyperref}
\usepackage[english,russian]{babel}
\usepackage[utf8x]{inputenc}
\usepackage{supertabular}
\usepackage{euscript}
\usepackage{textcomp}
\usepackage{graphicx}
\usepackage{amssymb}
\usepackage{amsmath}
\usepackage{amsthm}
\usepackage{cite}
\usepackage{agda}
\usepackage{ucs}

\textheight=650pt
\topmargin=-40pt

\begin{document}

  \begin{titlepage} 
     \begin{center}
        \textbf{\Large САНКТ-ПЕТЕРБУРГСКИЙ \\ ГОСУДАРСТВЕННЫЙ УНИВЕРСИТЕТ} \\[1.0cm]
        \textbf{\large Математико-Механический факультет} \\[0.2cm]
        \textbf{\large Кафедра Cистемного Программирования}\\[3.5cm]

        \textbf{\LARGE Использование proof assistants для описания операционных семантик}\\[1.0cm]
        \textbf{\Large Курсовая работа студента 445 группы} \\[0.2cm]
        \textbf{\Large Тарана Кирилла Сергеевича} \\[3.5cm]

        \begin{flushright} \large \emph{Научный руководитель:} \\ к.ф.-м.н., доцент    \textsc{Булычев Д. Ю.} \end{flushright}
        \begin{flushright} \large \emph{Заведующий кафедрой:}  \\ д.ф.-м.н., профессор \textsc{Терехов А. Н.} \end{flushright}
        \vfill 

        {\large {Санкт-Петербург}} \par
        {\large {2013 г.}}
     \end{center} 
  \end{titlepage}

  \tableofcontents
  \newpage

  \section{Введение}
  \subsection{Предметная область}
  \section{Обзор proof assistants}
  \subsection{Зависимые типы}
  \subsection{Соответствие Карри-Говарда}
  \section{Основная часть}
  \section{Заключение}

  % test
  \cite{Bove:2009:DTW:1614478.1614480}

  \subsection{Список литературы}
  \bibliographystyle{gost780s}
  \bibliography{literature}

\end{document}
